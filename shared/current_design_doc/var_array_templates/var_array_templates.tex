\documentclass[11pt]{report}

\usepackage{epsf,amsmath,amsfonts}
\usepackage{graphicx}
\usepackage{listings}
\usepackage{color}

\definecolor{gray}{rgb}{0.4,0.4,0.4}
\definecolor{darkblue}{rgb}{0.0,0.0,0.6}
\definecolor{cyan}{rgb}{0.0,0.6,0.6}

\lstset{
	basicstyle=\ttfamily,
		columns=fullflexible,
		showstringspaces=false,
		commentstyle=\color{gray}\upshape
}

\lstdefinelanguage{XML}
{
	morestring=[b]",
	morestring=[s]{>}{<},
	morecomment=[s]{<?}{?>},
	stringstyle=\color{black},
	identifierstyle=\color{darkblue},
	keywordstyle=\color{cyan},
	morekeywords={xmlns,version,type,name,type,dimensions,name\_in\_code,units,description,template,suffix,streams,units\_mod,description\_mod,persistence}% list your attributes here
}

\begin{document}

\title{Variable Array Templates: \\
Requirements and Design}
\author{MPAS Development Team}

\maketitle
\tableofcontents

%-----------------------------------------------------------------------

\chapter{Summary}

This document introduces variable array templates. Variable array templates are
an easy way for a developer to define a variable array container with
constituent variables once, in a template format, and later reference that
template multiple times.

The benefit of doing this is a single variable array can be maintained and
multiple instances of it can be used, thus keeping the total number of
constituent variables consistent across multiple variable arrays.


%-----------------------------------------------------------------------

\chapter{Requirements}

\section{Requirement: Templated variable arrays}
Date last modified: 10/08/2013 \\
Contributors: ( Doug Jacobsen ) \\

Templates are allowed to be defined for variable arrays, and variable arrays
are allowed to use these templates.

%-----------------------------------------------------------------------
\subsection{Motivation and Use Example}
As an example, the ocean core has a variable array called tracers with
constituents consisting of temperature, salinity, and a tracer with value of 1
for testing purposes. This variable array is duplicated in multiple places as
several quantities are computed using each of these tracers, including their
state information, tendency information, and surface fluxes as some examples.

Currently, in each var\_struct, the tracers variable array is defined as follows:

{\scriptsize
\begin{lstlisting}[language=XML]
<var_array name="tracers" type="real" dimensions="nVertLevels nCells Time">
	<var name="temperature" units="deg C" description="Potential temperature"/>
	<var name="salinity" units="PSU" description="Salinity"/>
	<var name="tracer1" units="percent" description="Tracer of 1 for testing"/>
</var_array>
\end{lstlisting}
}

One of the difficulties in maintaining this, is if a new tracer needs to be
added (for example, CO2 for biogeochemistry purposes), it has to be added to
each of these variable arrays to keep the constituent dimension size consistent
across all of the variable arrays.

When using templated variable arrays, these definitions would change to:

{\scriptsize
\begin{lstlisting}[language=XML]
<var_array_template name="tracers" type="real" dimensions="nVertLevels nCells Time">
	<var name="temperature" units="deg C" description="Potential temperature"/>
	<var name="salinity" units="PSU" description="Salinity"/>
	<var name="tracer1" units="PSU" description="Salinity"/>
</var_array_template>

<var_struct name="state" time_levs="2">
	<var_array name="tracers" template="tracers" streams="io"/>
	<var_array name="tracersSurfaceValues" template="tracers" suffix="SurfaceValues"
			   description_mod="extrapolated to the ocean surface" streams="o"/>
</var_struct>
<var_struct name="tend" time_levs="0">
	<var_array name="tracers" template="tracers" suffix="Tendencies"
			   units_mod="m s^{-1}" description_mod="tendencies"/>
</var_struct>
\end{lstlisting}
}

\chapter{Design and Implementation}

\section{Implementation: Variable array templates}
Date last modified: 10/08/2013
Contributors: ( Doug Jacobsen ) \\


Registry needs to be modified to allow templates to be defined within a
Registry.xml file. The new element type will be:
{\scriptsize
\begin{lstlisting}[language=XML]
<var_array_template name="name" type="real" dimensions="dims">
	<var name="name" name_in_code="name_in_code" units="units"
		 description="details"/>
</var_array_template>
\end{lstlisting}
}

The variable array templates live at the same level as the variable structures.

This allows a template to define multiple variables as constituent variables.
The name of the template will be used in variable arrays to reference this
template. A reference to a template will look like:

{\scriptsize
\begin{lstlisting}[language=XML]
<var_array name="name" template="template" suffix="Suffix"
		   units_mod="unit modifer" description_mod="description modifier"
		   streams="streams" persistence="persistence"/>
\end{lstlisting}
}

In this case, name will be the name of the variable array in the code, template
is the name of the template to expand within the variable array, suffix will be
appended to all of the constituent variables in the name field (not the
name\_in\_code field if it's specified), units\_mod will be appended to the
units field, description\_mod will be appended to the description field,
streams sets the default stresm for all constituents (in the case of a
templated variable array these cannot be overridden), and persistence sets the
persistence of this particular instance of the variable array.

Following the developer guide lines listed in our developers guide, suffix
should always be mixed case, and start with a capital letter. Also, only one
instance of a templated var\_array with a particular suffix can exist within
each var\_struct. This should 

When registry is parsing the Registry.xml if the template attribute is
specified on a variable array instance, registry searches for a matching
variable array template. If a matching template is found, the constituents are
expanded within the variable array. If a template is not found, registry
attemps to use the variables defined within the variable array in Registry.xml
if there are any. Note: if a template is not found, the resulting variable
array might not have any constituent variables. This should be checked in a
validator at some point.

\subsection{Attributes}
This project adds new attributes to the var\_array constructs in Registry.xml
in addition to defining the new var\_array\_template construct. Below are
tables to help describe these:

\begin{table}
	\begin{tabular}{| c | c |}
		\hline
		template & Defines the template to expand for this variable array. Needs to be the "name" attribute of a var\_array\_template. \\
		\hline
		suffix & Defines the suffix for the constituents contained in this instance of the var\_array. These are appended to the end of the constituent variables, but only in output files. \\
		\hline
		units\_mod & Defines the modifer for the units defined in the template. These are appended to the end of the templated units for each constituent. \\
		\hline
		description\_mod & Defines the modifer for the description defined in the template. These are appended to the end of the templated description for each constituent. \\
		\hline
	\end{tabular}
	\caption{New attributes for var\_array}
\end{table}

\begin{table}
	\begin{tabular}{| c | c |}
		\hline
		name & Defines the name of the template. Used in the template attribute for var\_array instances. \\
		\hline
		type & Defines the type of all instances of this template. \\
		\hline
		dimensions & Defines the default dimensions for all instances of this template. Dimensions can be overwritten on a per-instance basis through the var\_array dimensions attribute. \\
		\hline
	\end{tabular}
	\caption{Attributes for var\_array\_template}
\end{table}

\subsection{Constituent and Array group indices}
When using a templated variable array, only one instance of the
"index\_constituent\_name" indices are written to each var\_struct that the
templated is used within. Similarly, only one instance of the
"array\_group\_start" and "array\_group\_end" indices are written to each
var\_stuct. 

This is the "constituent\_name" and "array\_group" portions of these indices
are defined by the template. Meaning, the suffix for a particular instance of
the var\_array is not used when referencing a constituent, or an array\_group.

%-----------------------------------------------------------------------

\chapter{Testing}

\section{Testing and Validation: Templated variable arrays}
Date last modified: 10/08/2013 \\
Contributors: ( Doug Jacobsen ) \\

After templated variable arrays are implemented in registry, a variable array
in one of the components will be replaced with a templated variable array.
Simulation results should be bit-identical to the case where the variable array
is explicitly defined.

%-----------------------------------------------------------------------

\end{document}
